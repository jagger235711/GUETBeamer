\documentclass[aspectratio=169,AutoFakeBold]{beamer}
\usepackage{ctex,hyperref}
\usepackage[T1]{fontenc}
\usepackage{latexsym,amsmath,xcolor,multicol,booktabs,calligra}
\usepackage{graphicx,pstricks,listings,stackengine,gbt7714,tikz}
% ------------------------------------------
%     标题页
% ------------------------------------------
\author{汇报人:巫文杰 \texorpdfstring{\quad}{} 指导教师:曹建宇}
\title{周工作汇报}
% \subtitle{毕业设计答辩}
\institute{计算机与信息安全学院}
\date{\today}
\usepackage{GUETBeamer}
% \logo{\includegraphics[width=0.2\linewidth]{Guet-tm.pdf}} % 每一页添加logo

\def\cmd#1{\texttt{\color{red}\footnotesize $\backslash$#1}}
\def\env#1{\texttt{\color{blue}\footnotesize #1}}
\definecolor{deepblue}{rgb}{0,0,0.5}
\definecolor{deepred}{rgb}{0.6,0,0}
\definecolor{deepgreen}{rgb}{0,0.5,0}
\definecolor{halfgray}{gray}{0.55}

\lstset{
    basicstyle=\ttfamily\small,
    keywordstyle=\bfseries\color{deepblue},
    emphstyle=\ttfamily\color{deepred},   
    stringstyle=\color{deepgreen},
    numbers=left,
    numberstyle=\small\color{halfgray},
    rulesepcolor=\color{red!20!green!20!blue!20},
    frame=shadowbox,
}

\graphicspath{{./picture/}} % 图片所在位置

% ------------------------------------
%     正文
% ------------------------------------
\begin{document}

\kaishu
\begin{frame}
    \titlepage
    \begin{figure}[htpb]
        \begin{center}
            \includegraphics[width=0.5\linewidth]{guet-3.pdf}
        \end{center}
    \end{figure}
\end{frame}

\begin{frame}
    \tableofcontents[sectionstyle=show,subsectionstyle=show/shaded/hide,subsubsectionstyle=show/shaded/hide]
    
\end{frame}


\section{课题背景}

\begin{frame}{用Beamer很高大上?}
    \begin{itemize}[<+-| alert@+>] % 当然,除了alert,手动在里面插 \pause 也行
        \item 大家都会\LaTeX{},好多学校都有自己的Beamer主题
        \item 中文支持请选择 Xe\LaTeX{} 编译选项
        \item Overleaf项目地址在\url{https://www.overleaf.com/latex/templates/GUET-beamer-theme/ybqzdsgvrfdq},可以直接使用
    \end{itemize}
\end{frame}


\section{研究现状}

\subsection{Beamer主题分类}

\begin{frame}
    \begin{itemize}
        \item 有一些 \LaTeX{} 自带的
        \item 有一些GUET的
        \item 本模板来源自 \newline \url{https://www.latexstudio.net/archives/4051.html}
        \item 但是最初的 \href{http://far.tooold.cn/post/latex/beamerdlut}{\color{purple}{link}}\cite{吕李娜2021基于生成对抗网络的灰度照片上色方法} 已经失效了
        \item 整体设计参考自[Trinkle23897 / THU-Beamer-Theme](https://github.com/Trinkle23897/THU-Beamer-Theme)
    \end{itemize}
\end{frame}

\section{研究内容}

\subsection{美化主题}

\begin{frame}{主题说明}
    \begin{itemize}
        \item 顶栏的小点变成一行而不是多行
        \item 中文采用楷书
        \item 更多该模板的功能可以参考 \url{https://www.latexstudio.net/archives/4051.html}
        \item 下面列举出了一些Beamer的用法,部分节选自 \url{https://tuna.moe/event/2018/latex/}
    \end{itemize}
\end{frame}

\subsection{如何更好地做Beamer}

\begin{frame}{Why Beamer}
    \begin{itemize}
        \item \LaTeX 广泛用于学术界,期刊会议论文模板
    \end{itemize}
    \begin{table}[h]
        \centering
        \begin{tabular}{c|c}
            Microsoft\textsuperscript{\textregistered}  Word & \LaTeX        \\
            \hline
            文字处理工具                                           & 专业排版软件        \\
            容易上手,简单直观                                        & 容易上手          \\
            所见即所得                                            & 所见即所想,所想即所得   \\
            高级功能不易掌握                                         & 进阶难,但一般用不到    \\
            处理长文档需要丰富经验                                      & 和短文档处理基本无异    \\
            花费大量时间调格式                                        & 无需担心格式,专心作者内容 \\
            公式排版差强人意                                         & 尤其擅长公式排版      \\
            二进制格式,兼容性差                                       & 文本文件,易读、稳定    \\
            付费商业许可                                           & 自由免费使用        \\
        \end{tabular}
    \end{table}
\end{frame}

\begin{frame}{排版举例}

    \begin{textbox}{无编号公式}
        \begin{equation*}
            J(\theta) = \mathbb{E}_{\pi_\theta}[G_t] = \sum_{s\in\mathcal{S}} d^\pi (s)V^\pi(s)=\sum_{s\in\mathcal{S}} d^\pi(s)\sum_{a\in\mathcal{A}}\pi_\theta(a|s)Q^\pi(s,a)
        \end{equation*}
    \end{textbox}
    \begin{textbox}{多行多列公式\footnotemark[1]}
        % 使用 & 分隔
        \begin{align}
            Q_\mathrm{target} & =r+\gamma Q^\pi(s^\prime, \pi_\theta(s^\prime)+\epsilon)  \\
            \epsilon          & \sim\mathrm{clip}(\mathcal{N}(0, \sigma), -c, c)\nonumber
        \end{align}
    \end{textbox}
    \footnotetext[1]{如果公式中有文字出现,请用 $\backslash$mathrm\{\} 或者 $\backslash$text\{\} 包含,不然就会变成 $clip$,在公式里看起来比 $\mathrm{clip}$ 丑非常多。}
\end{frame}


\begin{frame}{如何使用块}
    \begin{block}{块的名称}
        \begin{itemize}
            \item A
            \item B
        \end{itemize}
    \end{block}
\end{frame}

\begin{frame}{如何使用定义、定理、引理、证明}

    
    \begin{define}[定义名称]
        定义内容
    \end{define}

    \begin{lem}[引理名称]
        引理内容
    \end{lem}

    \begin{thm}[定理名称]
        定理内容(这里的定义、引理、定理分章节自动标号)
    \end{thm}

    \begin{proof}
        证明内容
    \end{proof}

\end{frame}

\begin{frame}
    \begin{textbox}{编号多行公式}
        \begin{multline}
            A=\lim_{n\rightarrow\infty}\Delta x\left(a^{2}+\left(a^{2}+2a\Delta x+\left(\Delta x\right)^{2}\right)\right.\label{eq:reset}\\
            +\left(a^{2}+2\cdot2a\Delta x+2^{2}\left(\Delta x\right)^{2}\right)\\
            +\left(a^{2}+2\cdot3a\Delta x+3^{2}\left(\Delta x\right)^{2}\right)\\
            +\ldots\\
            \left.+\left(a^{2}+2\cdot(n-1)a\Delta x+(n-1)^{2}\left(\Delta x\right)^{2}\right)\right)\\
            =\frac{1}{3}\left(b^{3}-a^{3}\right)
        \end{multline}
    \end{textbox}
\end{frame}

\begin{frame}{图形与分栏}
    % From thuthesis user guide.
    \begin{minipage}[c]{0.3\linewidth}
        \psset{unit=0.8cm}
        \begin{pspicture}(-1.75,-3)(3.25,4)
            \psline[linewidth=0.25pt](0,0)(0,4)
            \rput[tl]{0}(0.2,2){$\vec e_z$}
            \rput[tr]{0}(-0.9,1.4){$\vec e$}
            \rput[tl]{0}(2.8,-1.1){$\vec C_{ptm{ext}}$}
            \rput[br]{0}(-0.3,2.1){$\theta$}
            \rput{25}(0,0){%
                \psframe[fillstyle=solid,fillcolor=lightgray,linewidth=.8pt](-0.1,-3.2)(0.1,0)}
            \rput{25}(0,0){%
                \psellipse[fillstyle=solid,fillcolor=yellow,linewidth=3pt](0,0)(1.5,0.5)}
            \rput{25}(0,0){%
                \psframe[fillstyle=solid,fillcolor=lightgray,linewidth=.8pt](-0.1,0)(0.1,3.2)}
            \rput{25}(0,0){\psline[linecolor=red,linewidth=1.5pt]{->}(0,0)(0.,2)}
            %           \psRotation{0}(0,3.5){$\dot\phi$}
            %           \psRotation{25}(-1.2,2.6){$\dot\psi$}
            \psline[linecolor=red,linewidth=1.25pt]{->}(0,0)(0,2)
            \psline[linecolor=red,linewidth=1.25pt]{->}(0,0)(3,-1)
            \psline[linecolor=red,linewidth=1.25pt]{->}(0,0)(2.85,-0.95)
            \psarc{->}{2.1}{90}{112.5}
            \rput[bl](.1,.01){C}
        \end{pspicture}
    \end{minipage}\hspace{1cm}
    \begin{minipage}{0.5\linewidth}
        \medskip
        %\hspace{2cm}
        \begin{figure}[h]
            \centering
            \includegraphics[height=.4\textheight]{dtmf.pdf}
        \end{figure}
    \end{minipage}
\end{frame}




\begin{frame}[fragile]{\LaTeX{} 常用命令}
    \begin{textbox}{命令}
        \centering
        \footnotesize
        \begin{tabular}{llll}
            \cmd{chapter}   & \cmd{section} & \cmd{subsection} & \cmd{paragraph}       \\
            章               & 节             & 小节               & 带题头段落                 \\\hline
            \cmd{centering} & \cmd{emph}    & \cmd{verb}       & \cmd{url}             \\
            居中对齐            & 强调            & 原样输出             & 超链接                   \\\hline
            \cmd{footnote}  & \cmd{item}    & \cmd{caption}    & \cmd{includegraphics} \\
            脚注              & 列表条目          & 标题               & 插入图片                  \\\hline
            \cmd{label}     & \cmd{cite}    & \cmd{ref}                                \\
            标号              & 引用参考文献        & 引用图表公式等                                  \\\hline
        \end{tabular}
    \end{textbox}
    \begin{textbox}{环境}
        \centering
        \footnotesize
        \begin{tabular}{lll}
            \env{table}   & \env{figure}    & \env{equation}    \\
            表格            & 图片              & 公式                \\\hline
            \env{itemize} & \env{enumerate} & \env{description} \\
            无编号列表         & 编号列表            & 描述                \\\hline
        \end{tabular}
    \end{textbox}
\end{frame}

\begin{frame}[fragile]{\LaTeX{} 环境命令举例}
    \begin{minipage}{0.5\linewidth}
        \begin{lstlisting}[language=TeX]
\begin{itemize}
  \item A \item B
  \item C
  \begin{itemize}
    \item C-1
  \end{itemize}
\end{itemize}
\end{lstlisting}
    \end{minipage}\hspace{1cm}
    \begin{minipage}{0.3\linewidth}
        \begin{itemize}
            \item A
            \item B
            \item C
                  \begin{itemize}
                      \item C-1
                  \end{itemize}
        \end{itemize}
    \end{minipage}
    \medskip
    \pause
    \begin{minipage}{0.5\linewidth}
        \begin{lstlisting}[language=TeX]
\begin{enumerate}
  \item 巨佬 \item 大佬
  \item 萌新
  \begin{itemize}
    \item[n+e] 瑟瑟发抖
  \end{itemize}
\end{enumerate}
\end{lstlisting}
    \end{minipage}\hspace{1cm}
    \begin{minipage}{0.3\linewidth}
        \begin{enumerate}
            \item 巨佬
            \item 大佬
            \item 萌新
                  \begin{itemize}
                      \item[n+e] 瑟瑟发抖
                  \end{itemize}
        \end{enumerate}
    \end{minipage}
\end{frame}

\begin{frame}[fragile]{\LaTeX{} 数学公式}
    \begin{columns}
        \begin{column}{.55\textwidth}
            \begin{lstlisting}[language=TeX]
$V = \frac{4}{3}\pi r^3$

\[
  V = \frac{4}{3}\pi r^3
\]

\begin{equation}
  \label{eq:vsphere}
  V = \frac{4}{3}\pi r^3
\end{equation}
\end{lstlisting}
        \end{column}
        \begin{column}{.4\textwidth}
            $V = \frac{4}{3}\pi r^3$
            \[
                V = \frac{4}{3}\pi r^3
            \]
            \begin{equation}
                \label{eq:vsphere}
                V = \frac{4}{3}\pi r^3
            \end{equation}
        \end{column}
    \end{columns}
    \begin{itemize}
        \item 更多内容请看 \href{https://zh.wikipedia.org/wiki/Help:数学公式}{\color{purple}{这里}}
    \end{itemize}
\end{frame}

\begin{frame}[fragile]
    \begin{columns}
        \column{.6\textwidth}
        \begin{lstlisting}[language=TeX]
    \begin{table}[htbp]
      \caption{编号与含义}
      \label{tab:number}
      \centering
      \begin{tabular}{cl}
        \toprule
        编号 & 含义 \\
        \midrule
        1 & 4.0 \\
        2 & 3.7 \\
        \bottomrule
      \end{tabular}
    \end{table}
    公式~(\ref{eq:vsphere}) 的
    编号与含义请参见
    表~\ref{tab:number}。
\end{lstlisting}
        \column{.4\textwidth}
        \begin{table}[htpb]
            \centering
            \caption{编号与含义}
            \label{tab:number}
            \begin{tabular}{cl}\toprule
                编号 & 含义  \\\midrule
                1  & 4.0 \\
                2  & 3.7 \\\bottomrule
            \end{tabular}
        \end{table}
        \normalsize 公式~(\ref{eq:vsphere})的编号与含义请参见表~\ref{tab:number}。
    \end{columns}
\end{frame}

\begin{frame}{作图}
    \begin{itemize}
        \item 矢量图 eps, ps, pdf
              \begin{itemize}
                  \item METAPOST, pstricks, pgf $\ldots$
                  \item Xfig, Dia, Visio, Inkscape $\ldots$
                  \item Matlab / Excel 等保存为 pdf
              \end{itemize}
        \item 标量图 png, jpg, tiff $\ldots$
              \begin{itemize}
                  \item 提高清晰度,避免发虚
                  \item 应尽量避免使用
              \end{itemize}
    \end{itemize}
    \begin{figure}[htpb]
        \centering
        \includegraphics[width=0.2\linewidth]{Guet-logo.pdf}
        \caption{这个校徽就是矢量图}
    \end{figure}
\end{frame}


\section{计划进度}
\begin{frame}{计划进度}
    \begin{itemize}
        \item 一月:完成文献调研\cite{张营营2018生成对抗网络模型综述}
        \item 二月:复现并评测各种Beamer主题美观程度
        \item 三、四月:美化GUET Beamer主题\cite{creswell2018generative}
        \item 五月:论文撰写
    \end{itemize}
\end{frame}


\section{参考文献}

\begin{frame}[allowframebreaks]{参考文献}
    \bibliography{ref}
    % \tiny\bibliographystyle{gbt7714-numerical}
    % 如果参考文献太多的话,可以像下面这样调整字体:
    % \tiny\bibliographystyle{alpha}
\end{frame}

% ------------------------------------------
%     新的章节结构
% ------------------------------------------
\section{汇报背景与目标}
\begin{frame}{课题背景与目标}
    \begin{itemize}[<+-| alert@+>]
        \item 导师布置任务:阅读三篇核心论文(A/B/C)
        \item 学习目标:
            \begin{itemize}
                \item 了解多模态的数据处理技术
                \item 了解多模态大模型的预训练方法
                \item 了解专为智能制造应用设计的多模态大模型系统
            \end{itemize}
    \end{itemize}
\end{frame}

\section{三篇论文概览}
\begin{frame}{论文核心内容分析}
    \begin{block}{论文A:Data processing techniques for modern multimodal models}
        \begin{itemize}
            \item 数据质量
            \item 数据数量
            \item 数据分布
            \item 数据安全
        \end{itemize}
    \end{block}

    \begin{block}{论文B:A Survey of Multimodal Large Language Model from A Data-centric Perspective}
        \begin{itemize}
            \item 亮点:从以数据为中心的视角下研究多模态大模型的预训练、数据准备以及数据集的评估方法
            \item 疑问:训练数据集的构建方法
        \end{itemize}
    \end{block}
\end{frame}

\section{查漏补缺与学习路径}
\begin{frame}{知识盲点与学习路径}
    \begin{columns}[T]
        \begin{column}{.5\textwidth}
            \begin{block}{关键知识盲点}
                \begin{itemize}
                    \item GAN的梯度传播机制
                    \item Lab色彩空间转换原理
                    \item 感知损失函数实现
                \end{itemize}
            \end{block}
        \end{column}
        \begin{column}{.5\textwidth}
            \begin{block}{学习资源}
                \begin{itemize}
                    \item 《深度学习》第9章
                    \item CS231n课程视频
                    \item OpenCV文档
                \end{itemize}
            \end{block}
        \end{column}
    \end{columns}

    \vspace{1em}
    \centering
    \begin{tikzpicture}[node distance=1.5cm]
        \node[draw] (start) {论文阅读};
        \node[draw, below left of=start] (math) {数学基础};
        \node[draw, below right of=start] (code) {代码实践};
        \draw[->] (start) -- (math);
        \draw[->] (start) -- (code);
    \end{tikzpicture}
\end{frame}


\section{知识框架总结}
\begin{frame}{知识体系构建}
    \centering
    \begin{tikzpicture}[node distance=2cm]
        \node[draw, rounded corners] (core) {核心论文};
        \node[draw, below left of=core] (math) {最优化理论};
        \node[draw, below right of=core] (dl) {深度学习};
        \draw[->] (core) -- (math) node[midway,left] {数学基础};
        \draw[->] (core) -- (dl) node[midway,right] {实现手段};
        \node[draw, below of=math] (gan) {GAN原理};
        \node[draw, below of=dl] (cv) {计算机视觉};
    \end{tikzpicture}
\end{frame}

\section{收获与不足}
\begin{frame}{阶段性成果}
    \begin{textbox}{已掌握技能}
        \begin{itemize}
            \item GAN网络结构与训练技巧
            \item 色彩空间转换原理(RGB/Lab)
            \item 图像质量评估指标(PSNR/SSIM)
        \end{itemize}
    \end{textbox}
    
    \begin{textbox}{待解决问题}
        \begin{itemize}
            \item 多模态融合策略
            \item 小样本条件下的模型泛化
        \end{itemize} 
    \end{textbox}
\end{frame}

\section{下一步计划}




\section{下一步计划}
\begin{frame}{后续研究规划}
    \begin{enumerate}
        \item \textbf{文献拓展}
            \begin{itemize}
                \item 精读CVPR最新图像上色论文
                \item 研究多模态融合前沿方法
            \end{itemize}
        \item \textbf{实践计划}
            \begin{itemize}
                \item 复现论文核心算法
                \item 构建自定义数据集
            \end{itemize}
        \item \textbf{理论深化}
            \begin{itemize}
                \item 学习流形学习理论
                \item 研究色彩空间映射数学基础
            \end{itemize}
    \end{enumerate}
\end{frame}

\begin{frame}
    \Background
    \begin{center}
        {\Huge\calligra 谢谢聆听}
    \end{center}
\end{frame}

\end{document}