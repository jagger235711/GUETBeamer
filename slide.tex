\documentclass[aspectratio=169,AutoFakeBold]{beamer}
\usepackage{ctex,hyperref}
\usepackage[T1]{fontenc}
\usepackage{latexsym,amsmath,xcolor,multicol,booktabs,calligra}
\usepackage{graphicx,pstricks,listings,stackengine,gbt7714,tikz}
% ------------------------------------------
%     标题页
% ------------------------------------------
\author{汇报人:巫文杰 \texorpdfstring{\quad}{} 指导教师:曹建宇}
\title{周工作汇报}
% \subtitle{毕业设计答辩}
\institute{计算机与信息安全学院}
\date{\today}
\usepackage{GUETBeamer}
% \logo{\includegraphics[width=0.2\linewidth]{Guet-tm.pdf}} % 每一页添加logo

\def\cmd#1{\texttt{\color{red}\footnotesize $\backslash$#1}}
\def\env#1{\texttt{\color{blue}\footnotesize #1}}
\definecolor{deepblue}{rgb}{0,0,0.5}
\definecolor{deepred}{rgb}{0.6,0,0}
\definecolor{deepgreen}{rgb}{0,0.5,0}
\definecolor{halfgray}{gray}{0.55}

\lstset{
    basicstyle=\ttfamily\small,
    keywordstyle=\bfseries\color{deepblue},
    emphstyle=\ttfamily\color{deepred},   
    stringstyle=\color{deepgreen},
    numbers=left,
    numberstyle=\small\color{halfgray},
    rulesepcolor=\color{red!20!green!20!blue!20},
    frame=shadowbox,
}

\graphicspath{{./picture/}} % 图片所在位置

% ------------------------------------
%     正文
% ------------------------------------
\begin{document}

\kaishu
\begin{frame}
    \titlepage
    \begin{figure}[htpb]
        \begin{center}
            \includegraphics[width=0.5\linewidth]{guet-3.pdf}
        \end{center}
    \end{figure}
\end{frame}

\begin{frame}
    \tableofcontents[sectionstyle=show,subsectionstyle=show/shaded/hide,subsubsectionstyle=show/shaded/hide]    
\end{frame}

% ------------------------------------------
%     新的章节结构
% ------------------------------------------
\section{汇报背景与目标}
\begin{frame}{课题背景与目标}
    \begin{itemize}[<+-| alert@+>]
        \item 导师布置任务:阅读三篇核心论文(A/B/C)
        \item 学习目标:
            \begin{itemize}
                \item 了解多模态的数据处理技术
                \item 了解多模态大模型的预训练方法
                \item 了解专为智能制造应用设计的多模态大模型系统
            \end{itemize}
    \end{itemize}
\end{frame}

\section{三篇论文概览}
\begin{frame}{论文核心内容分析}
    \begin{block}{论文A:Data processing techniques for modern multimodal models}
        \begin{itemize}
            \item 数据质量
            \item 数据数量
            \item 数据分布
            \item 数据安全
        \end{itemize}
    \end{block}

    \begin{block}{论文B:A Survey of Multimodal Large Language Model from A Data-centric Perspective}
        \begin{itemize}
            \item 亮点:从以数据为中心的视角下研究多模态大模型的预训练、数据准备以及数据集的评估方法
            \item 疑问:训练数据集的构建方法
        \end{itemize}
    \end{block}
\end{frame}

\section{查漏补缺与学习路径}
\begin{frame}{知识盲点与学习路径}
    \begin{columns}[T]
        \begin{column}{.5\textwidth}
            \begin{block}{关键知识盲点}
                \begin{itemize}
                    \item GAN的梯度传播机制
                    \item Lab色彩空间转换原理
                    \item 感知损失函数实现
                \end{itemize}
            \end{block}
        \end{column}
        \begin{column}{.5\textwidth}
            \begin{block}{学习资源}
                \begin{itemize}
                    \item 《深度学习》第9章
                    \item CS231n课程视频
                    \item OpenCV文档
                \end{itemize}
            \end{block}
        \end{column}
    \end{columns}

    \vspace{1em}
    \centering
    \begin{tikzpicture}[node distance=1.5cm]
        \node[draw] (start) {论文阅读};
        \node[draw, below left of=start] (math) {数学基础};
        \node[draw, below right of=start] (code) {代码实践};
        \draw[->] (start) -- (math);
        \draw[->] (start) -- (code);
    \end{tikzpicture}
\end{frame}

\section{知识框架总结}
\begin{frame}{知识体系构建}
    \centering
    \begin{tikzpicture}[node distance=2cm]
        \node[draw, rounded corners] (core) {核心论文};
        \node[draw, below left of=core] (math) {最优化理论};
        \node[draw, below right of=core] (dl) {深度学习};
        \draw[->] (core) -- (math) node[midway,left] {数学基础};
        \draw[->] (core) -- (dl) node[midway,right] {实现手段};
        \node[draw, below of=math] (gan) {GAN原理};
        \node[draw, below of=dl] (cv) {计算机视觉};
    \end{tikzpicture}
\end{frame}

\section{收获与不足}
\begin{frame}{阶段性成果}
    \begin{textbox}{已掌握技能}
        \begin{itemize}
            \item GAN网络结构与训练技巧
            \item 色彩空间转换原理(RGB/Lab)
            \item 图像质量评估指标(PSNR/SSIM)
        \end{itemize}
    \end{textbox}
    
    \begin{textbox}{待解决问题}
        \begin{itemize}
            \item 多模态融合策略
            \item 小样本条件下的模型泛化
        \end{itemize} 
    \end{textbox}
\end{frame}

\section{下一步计划}
\begin{frame}{后续研究规划}
    \begin{enumerate}
        \item \textbf{文献拓展}
            \begin{itemize}
                \item 精读CVPR最新图像上色论文
                \item 研究多模态融合前沿方法
            \end{itemize}
        \item \textbf{实践计划}
            \begin{itemize}
                \item 复现论文核心算法
                \item 构建自定义数据集
            \end{itemize}
        \item \textbf{理论深化}
            \begin{itemize}
                \item 学习流形学习理论
                \item 研究色彩空间映射数学基础
            \end{itemize}
    \end{enumerate}
\end{frame}

\begin{frame}
    \Background
    \begin{center}
        {\Huge\calligra 谢谢聆听}
    \end{center}
\end{frame}

\section{参考文献}


\end{document}