\documentclass[aspectratio=169,AutoFakeBold]{beamer}
\usepackage{ctex,hyperref}
\usepackage{bookmark}
\usepackage[T1]{fontenc}
\usepackage{latexsym,amsmath,xcolor,multicol,booktabs,calligra}
\usepackage{graphicx,pstricks,listings,stackengine,gbt7714,tikz}
% ------------------------------------------
%     标题页
% ------------------------------------------
\author{汇报人:巫文杰 \texorpdfstring{\quad}{} 指导教师:曹建宇}
\title{周工作汇报}
% \subtitle{毕业设计答辩}
\institute{计算机与信息安全学院}
\date{\today}
\usepackage{GUETBeamer}
% \logo{\includegraphics[width=0.2\linewidth]{Guet-tm.pdf}} % 每一页添加logo

\def\cmd#1{\texttt{\color{red}\footnotesize $\backslash$#1}}
\def\env#1{\texttt{\color{blue}\footnotesize #1}}
\definecolor{deepblue}{rgb}{0,0,0.5}
\definecolor{deepred}{rgb}{0.6,0,0}
\definecolor{deepgreen}{rgb}{0,0.5,0}
\definecolor{halfgray}{gray}{0.55}

\lstset{
    basicstyle=\ttfamily\small,
    keywordstyle=\bfseries\color{deepblue},
    emphstyle=\ttfamily\color{deepred},   
    stringstyle=\color{deepgreen},
    numbers=left,
    numberstyle=\small\color{halfgray},
    rulesepcolor=\color{red!20!green!20!blue!20},
    frame=shadowbox,
}

\graphicspath{{./picture/}} % 图片所在位置

% ------------------------------------
%     正文
% ------------------------------------
\begin{document}

\kaishu
\begin{frame}
    \titlepage
    \begin{figure}[htpb]
        \begin{center}
            \includegraphics[width=0.5\linewidth]{guet-3.pdf}
        \end{center}
    \end{figure}
\end{frame}

\begin{frame}
    \tableofcontents[sectionstyle=show,subsectionstyle=show/shaded/hide,subsubsectionstyle=show/shaded/hide]    
\end{frame}

% ------------------------------------------
%     新的章节结构  
% ------------------------------------------
\section{汇报背景与目标}
\begin{frame}{Introduction}
    \small
    在暑期学习中,我根据导师的要求研读了三篇核心论文(A / B / C)。这些论文分别涉及多模态数据处理技术、多模态大模型的预训练方法、数据准备以及数据集的评估方法,以及面向智能制造的多模态大模型系统。为了深入理解相关内容,我在阅读的过程中系统梳理了深度学习的基本概念,学习了经典深度学习方法,进一步关注了深度学习的相关知识。通过这种“论文研读 + 补充学习”的方式,我初步梳理了后续的学习方向。
\end{frame}
\begin{frame}{课题背景与目标}
    \begin{itemize}
        \item 导师布置任务:阅读三篇核心论文(A/B/C)
        \item 学习目标:
            \begin{itemize}
                \item 了解深度学习基础知识
                \item 了解多模态的数据处理技术
                \item 了解多模态大模型的预训练方法、数据准备以及数据集的评估方法
                \item 了解专为智能制造应用设计的多模态大模型系统
            \end{itemize}
    \end{itemize}
\end{frame}

\section{三篇论文概览}
\begin{frame}{论文核心内容分析}
    % 第1步显示:仅论文A
    \only<1>{
        \begin{block}{论文A:Data processing techniques for modern multimodal models}
            \begin{itemize}
                \item 数据质量
                \item 数据数量
                \item 数据分布
                \item 数据安全
            \end{itemize}
        \end{block}
    }
    
    % 第2步显示:仅论文B
    \only<2>{
        \begin{block}{论文B:A Survey of Multimodal Large Language Model from A Data-centric Perspective}
            \begin{itemize}
                \item 亮点:从以数据为中心的视角下研究多模态大模型的预训练、数据准备以及数据集的评估方法
                \item 疑问:训练数据集的构建方法
            \end{itemize}
            ----------------------------
            \begin{itemize}
                \item “如何收集,选择和管理MLLM的数据”
                \item “数据如何影响MLLM的性能” 
                \item “如何评估MLLM的数据” 
            \end{itemize}
        \end{block}
    }
    
    % 第3步显示:仅论文C
    \only<3>{
        \begin{block}{论文C:A Multimodal Large Language Model Framework for Intelligent Perception and Decision-Making in Smart Manufacturing}
            \begin{itemize}
                \item 一种统一的方法,用于表示不同的数据类型,动态语义令牌化,以进行更好的数据处理,跨模态的强大对齐策略
                \item 一种实用的两阶段训练方法,涉及初始大规模预处理以及以后进行特定任务进行微调。
                \item 引入了一种基于Transformer的新型模型,用于生成图像和文本,从而显着提高了实时决策能力。
            \end{itemize}
        \end{block}
    }
\end{frame}

\section{查漏补缺与学习路径}
\begin{frame}{知识盲点与学习路径}
    \begin{columns}[T]
        \begin{column}{.5\textwidth}
            \begin{block}{关键知识盲点}
                \begin{itemize}
                    \item 包括Transformer在内的经典网络结构与原理
                    \item 数据预处理方法
                    \item 多模态数据融合策略
                    \item 多模态大模型训练技巧
                \end{itemize}
            \end{block}
        \end{column}
        \begin{column}{.5\textwidth}
            \begin{block}{学习资源}
                \begin{itemize}
                    \item 《动手学深度学习》李沐
                    \item CS 329P: Practical Machine Learning  斯坦福课程
                \end{itemize}
            \end{block}
        \end{column}
    \end{columns}
\end{frame}

% \section{收获与不足}
% \begin{frame}{阶段性成果}
%     \begin{textbox}{已掌握技能}
%         \begin{itemize}
%             \item GAN网络结构与训练技巧
%             \item 色彩空间转换原理(RGB/Lab)
%             \item 图像质量评估指标(PSNR/SSIM)
%         \end{itemize}
%     \end{textbox}
    
%     \begin{textbox}{待解决问题}
%         \begin{itemize}
%             \item 多模态融合策略
%             \item 小样本条件下的模型泛化
%         \end{itemize} 
%     \end{textbox}
% \end{frame}

\section{下一步计划}
\begin{frame}{后续研究规划}
    \begin{enumerate}
        \item \textbf{知识拓展}
            \begin{itemize}
                \item 学习相关课程如 CS 329P
                \item 了解多模态融合前沿方法
            \end{itemize}
        \item \textbf{实践计划}
            \begin{itemize}
                \item 复现课程代码
            \end{itemize}
        \item \textbf{理论深化}
            \begin{itemize}
                \item 学习经典网络架构
                \item 补充知识盲区
            \end{itemize}
    \end{enumerate}
\end{frame}


\section{Q\&A}
\begin{frame}{Q\&A}
    \begin{itemize}
            \item 怎么找到合适的论文?
            \item 怎么样读论文?
            \item 怎么样做笔记?
            \item 怎么样写论文?
        \end{itemize} 
\end{frame}

\begin{frame}
    \Background
    \begin{center}
        {\Huge\calligra 谢谢聆听}
    \end{center}
\end{frame}

\end{document}
